\section{Contour Integration}

Arguably, the most important parts of complex analysis address the integration of complex-valued functions along curves, especially holomorphic ones. We're going to use contour integrals to establish some deeper results about holomorphic functions, some of which might be extremely surprising!

\subsection{What is a Contour Integral?}
But first -- what is a contour integral? If you're familiar with integrating along curves in $\RR^2$, you won't be surprised to see that we do essentially the same thing in $\CC$.

First, what kinds of curves can we integrate along? Generally, we want to be able to integrate along smooth paths in $\CC$, but the definition of path we discussed in the previous section only has the ``continuous'' restriction. This means that if we were to stick only to paths, we would have to figure out how to deal with self-similar fractal curves and also space-filling Peano curves, and we really don't want to integrate along those. Stein-Shakarchi avoids the problem of dealing with this by essentially only considering curves that are okay for us integrate along to have only a finite number of non-smooth points, which is pretty much good enough for any reasonable purpose. We will actually go a smidge more general and talk instead about curves with finite arc length that can be approximated by line segments, which are called \textit{rectifiable curves}.

Formally, we define a \textbf{rectifiable curve} in $\CC$ as a continuous function $\gamma : [0,1] \to \CC$ such that there is a continuous bijection $\vphi : [0,1] \to [0,1]$ such that $\vphi \circ \gamma$ is Lipschitz. The Lipschitz condition will be instrumental in showing that the length of such a curve will be finite when we integrate along it.\footnote{This source is what I'm referencing for this statement: \href{https://encyclopediaofmath.org/wiki/Rectifiable_curve}{Rectifiable curve}} Really, we should be considering equivalence classes of continuous functions $\gamma : [0,1] \to \CC$, such that $\gamma$ is in the same equivalence class as some other continuous $\gamma' : [0,1] \to \CC$ if we have a continuous bijection $\psi : [0,1] \to [0,1]$ such that $\gamma' = \psi \circ \gamma$. These equivalence classes are the actual curves $C$, which can be parameterized in possibly many ways (here, the $\gamma$ and $\gamma'$). We also call these rectifiable curves \textbf{contours}.

Okay, so let's suppose we have a contour $C$ as above, parameterized by some $\gamma: [0,1] \to \CC$. Then, we define the \textbf{contour integral} of $\int_C f(z) \, dz$ some complex-valued function $f : \CC \to \CC$ along $C$ by
\[ \int_C f(z) \, dz = \lim_{n\to \infty} \sum_{k=1}^n f \left(\gamma\left(\frac k n \right)\right)  \cdot \left[\gamma \left(\frac k n\right) - \gamma\left(\frac{k-1} n \right) \right] = \int_0^1 f(\gamma(t)) \gamma'(t) \, dt .\]
This should look really similar to the Riemann sum that you're used to, and you'll study this more in 2240 and get a glimpse as to why this real-valued integral converges when the curve is appropriately parameterized. (If you're concerned about the complex values of $f$, we could separate this into the real and imaginary parts and treat them separately, so this really reduces to the real-integral case.)

We can now compute integrals via parameterization explicitly, but this isn't really going to be useful for us to develop, and often exercises for this involve integrating non-holomorphic functions, which is really similar to just computing a line integral and isn't going to be useful for us (as this method will not be used to actually compute integrals, 99\% of the time). Instead, let's start by developing some tools. Like in single-variable calculus, we have a \textbf{Fundamental Theorem of Contour Integrals}, which is a complex version of the Fundamental Theorem of Calculus:
\begin{theorem}
  Let $f : U \to \CC$ be a continuous function and $F : U \to \CC$ be holomorphic such that $F'(z) = f(z)$. Then, for a curve $\gamma$ that begins at $z_1$ and ends at $z_2$, we have that \[\int_\gamma f(z) \, dz = F(z_2) - F(z_1).\]
\end{theorem}
The proof of this is essentially the same as that as the Fundamental Theorem of Calculus in $\RR$.

We will also be very interested in integrating around closed loops in $\CC$. A contour is \textbf{closed} if its endpoints are the same, i.e. if $\gamma : [0,1] \to \CC$ defines a contour $C$, then $C$ is closed if $\gamma(0) = \gamma(1)$, and a contour that is not self-intersecting is called \textbf{simple}. An example of a simple closed contour is the unit circle centered at the origin, which may be oriented in one of two different ways. We call the counterclockwise orientation of any loop the \textbf{positive} orientation by convention, and the clockwise orientation of a loop the \textbf{negative} orientation. Integrating along a closed loop in opposite orientations gives the negative answer.

For instance, we can integrate the function $f(z) = \frac 1z$ around the unit circle counterlockwise, along contour $\gamma^+$ parameterized by $\gamma^+ : [0, 2\pi] \to \CC$, $\gamma^+(t) = e^{it}$ (we can easily rescale the parameterization to be on $[0,1]$ if desired):
\[ \oint_{\gamma^+} \frac 1z \, dz = \int_0^{2\pi} \frac{i e^{it}}{e^{it}} \, dt = 2\pi i.\]
Or, we can integrate the same function clockwise around the unit circle, along contour $\gamma^-$ parameterized by $\gamma^- : [0, 2\pi] \to \CC$, $\gamma^-(t) = e^{-it}$:
\[ \oint_{\gamma^-} \frac 1z \, dz = \int_0^{2\pi} \frac{-i e^{-it}}{e^{-it}} \, dt = -2\pi i.\]
Note that we denote integrating around a closed contour with the $\oint$ integration symbol as opposed to the regular $\int$ sign, as the circle denotes that we're integrating around a loop.

We can apply the Fundamental Theorem of Calculus above for contour integrals to get a useful-ish result for closed contours:
\begin{corollary}
  If $f : U \to \CC$ is continuous and there exists some $F : U \to \CC$ such that $F$ is holomorphic on $U$ and $F'(z) = f(z)$, then if $C$ is a closed contour contained in $U$, then \[\oint_C f(z) \, dz = 0.\]
\end{corollary}
Note that this result does NOT apply to our example above! If we wanted to construct the antiderivative of $\frac 1z$ as $\Log z$, remember that the $\Log$ function has a branch cut along the negative real axis and a branch point at 0! There does not exist a domain on which this antiderivative is holomorphic that contains the domain, as the branch cut must cross the path.

\subsection{Integral Bounds}
Often, in the course of proofs or calculations we will need to construct bounds on integrals, so we will develop a couple of cool tricks here.

The most important of these is the \textbf{ML Lemma}:
\begin{theorem}[ML Lemma]
  Suppose $C$ is a rectifiable contour with length $L$ in some open $U \subseteq \CC$, and let the maximum modulus of some continuous function $f : U \to \CC$ on $U$ be $M$. Then, \[\abs{\int_C f(z) \, dz} \leq ML .\]
\end{theorem}
Note that the maximum exists as the contour is a compact subset of $\CC$.

The proof of this is fairly straightforward, using the triangle inequality and the limit definition of the contour integral:
\begin{proof}
  \begin{align*}
    \abs{\int_C f(z) \, dz}
     & = \abs{ \lim_{n\to \infty} \sum_{k=1}^n f \left(\gamma\left(\frac k n \right)\right)  \cdot \left[\gamma \left(\frac k n\right) - \gamma\left(\frac{k-1} n \right) \right] }           \\
     & \leq  \lim_{n\to \infty} \sum_{k=1}^n \abs{ f \left(\gamma\left(\frac k n \right)\right)}  \cdot \abs{\left[\gamma \left(\frac k n\right) - \gamma\left(\frac{k-1} n \right) \right] } \\
     & \leq M \lim_{n\to \infty} \sum_{k=1}^n \abs{\left[\gamma \left(\frac k n\right) - \gamma\left(\frac{k-1} n \right) \right] }                                                           \\
     & \leq M L
  \end{align*}
\end{proof}
This is an incredibly useful bound, and it's used in many proofs and computations.

Another bound that's useful mostly for computational purposes is \textbf{Jordan's Lemma}, which applies particularly to semicircular contours in the upper half-plane:
\begin{theorem}[Jordan's Lemma]
  Let $C_R$ be the contour $\braces{R e^{it} : t \in [0, \pi]}$ for $R > 0$ real, and let $f : \CC \to \CC$ be continuous on $C_R$ with $|f(z)|$ bounded by $M_R$ on $C_R$. Then, \[ \abs{\int_{C_R} f(z) e^{iz} \, dz} \leq M_R \pi  .\]
\end{theorem}
\begin{proof}
  \begin{align*}
    \abs{\int_{C_R} f(z) \, dz}
     & = \abs{\int_0^\pi f(Re^{it}) e^{i Re^{it}} \cdot iR e^{it} \, dt} = \abs{\int_0^\pi f(Re^{it}) e^{i R(\cos t + i \sin t)} \cdot iR e^{it} \, dt} \\
     & \leq \int_0^\pi \abs{f(Re^{it}) e^{i R\cos t} \cdot e^{-R \sin t} \cdot iR e^{it}} \, dt                                                         \\
     & = R \int_0^\pi \abs{f(Re^{it})}  \cdot \abs{e^{-R \sin t}} \, dt                                                                                 \\
     & \leq M_R R \int_0^\pi \abs{e^{-R \sin t}} \, dt = 2M_R R \int_0^{\frac \pi 2} \abs{e^{-R \sin t}} \, dt
  \end{align*}
  It suffices to find a good bound for $e^{-R \sin t}$. Note that since $\sin t \geq \frac 2 \pi t$ on $[0, \frac \pi 2]$, that therefore $e^{-R \sin t} \leq e^{-\frac 2 \pi t}$. We can thus finish up the bounding:
  \begin{align*}
    \abs{\int_{C_R} f(z) \, dz} \leq 2M_R R \int_0^{\frac \pi 2} e^{-R \sin t} \, dt \leq 2 M_R R \int_0^{\frac \pi 2} e^{-\frac{2R}{\pi} t} \, dt = \frac{2 M_R R \pi}{2R} \left( 1 - e^{-R} \right) = M_R \pi \left( 1 - e^{-R} \right) \leq M_R \pi \\
  \end{align*}
\end{proof}



\subsection{The Cauchy-Goursat Theorem}
The theorem that follows is possibly the most important theorem in complex analysis. It's the first of the three ``miracles'' of smooth functions that
\begin{theorem}[Cauchy-Goursat]
  Let $U$ be an open and \textbf{simply connected} subset of $\CC$ and $f : U \to \CC$ be holomorphic on $U$. Then, for any simple closed contour $C$ contained in $U$, we have that \[\oint_C f(z) \, dz = 0.\]
\end{theorem}

This is a pretty hard theorem to prove rigorously... common attempts involve using Green's Theorem, but this assumes that $f$ has to have continuous real and imaginary partials with respect to the real/imaginary parts of the inputs, which is... not necessarily known? It actually turns out this is a little circular, but it's not a bad method of verifying the theorem, I guess.

The proper way was created by Goursat, and his proof is the one I will sketch out here.


\subsubsection{A Note on Homotopy}


\subsection{Integral Formulas}
As a result of Cauchy-Goursat, we deduce the following result --
\begin{theorem}[Cauchy's Integral Formula]
  Let $U$ be an open and simply connected subset of $\CC$ and $f : U \to \CC$ be holomorphic on $U$. Then, for any simple closed contour $C$ and some $z_0$ in the interior of $C$, we have that \[f(z_0) = \frac 1{2\pi i}\oint_C \frac{f(z)}{z-z_0} \, dz. \]
\end{theorem}

\begin{theorem}[Cauchy's Integral Formula For Derivatives]
  Let $U$ be an open and simply connected subset of $\CC$ and $f : U \to \CC$ be holomorphic on $U$. Then, for any simple closed contour $C$ and some $z_0$ in the interior of $C$, we have that \[f^{(n)}(z_0) = \frac {n!}{2\pi i}\oint_C \frac{f(z)}{(z-z_0)^{n+1}} \, dz. \]
\end{theorem}

\subsection{Miracles (Consequences of CIF and Friends)}
%% Taylor's Theorem
\begin{theorem}[Taylor's Theorem]
  Suppose $f : U \to \CC$ is holomorphic on open set $U \subseteq \CC$. Then, for any $z_0 \in U$, there exists an open disk $D$ whose closure is contained in $U$ where $f$ has a power series expansion centered at $z_0$:
  \[ f(z) = \sum_{k = 0}^\infty a_k (z-z_0)^k, \quad a_k = \frac{f^{(k)}(z_0)}{k!}.\]
  This power series expansion converges to $f$ for every point $z \in D$.
\end{theorem}

%% INFINITE DIFFERENTIABILTY WOOOOOO
\begin{corollary}[Infinite differentiability]
  A function $f : U \to \CC$ holomorphic in some open set $U \subseteq \CC$ is infinitely differentiable on $U$.
\end{corollary}
% uses CIFFD
As such, we have the corollary that holomorphic functions are therefore analytic, by Taylor's Theorem (the complex version). In particular, this means that holomorphic $\Leftrightarrow$ analytic, so we will use the words interchangably from here on out.

%% Liouville's Theorem 
\begin{theorem}[Liouville's Theorem]
  A function that is entire and bounded in magnitude is constant.
\end{theorem}

%% Corollary: Fundamental Theorem of Algebra (not stupid)
\begin{corollary}[The Fundamental Theorem of Algebra, Again (Stein-Shakarchi)]
  % suppose for sake of contradiction p(z) has no zeroes
  % then 1/p is entire
  % can also bound 1/p 
  % therefore 1/p is constant, but this is literally not the case
  % EZ
\end{corollary}

%% Morera's Theorem
Morera's Theorem is like a converse to Cauchy-Goursat -- it tells us when a function is holomorphic based on its loop integrals:
\begin{theorem}[Morera's Theorem]
  Let $f : D \to \CC$ be continuous in some open set $D \subseteq \CC$ such that for every simple closed rectifiable contour $C$ contained in $D$, $\oint_C f(z) \, dz = 0$. Then $f$ is holomorphic on $D$.
\end{theorem}

%% Analytic continuation (follows Stein-Shakarchi)
The third miracle as advertised in Stein-Shakarchi is the idea of \textbf{analytic continuation}, where one may construct extensions of analytic functions.
\begin{theorem}[Continuation of Zero]
  Suppose $f$ is a holomorphic function on some open connected $U \subseteq \CC$ that vanishes on a sequence of distinct points $\braces{z_k}_{k=1}^\infty$ such that $z_k \in U$. If this sequence $\braces{z_k}$ has a limit point, then $f = 0$ on $U$.
\end{theorem}

\begin{corollary}[Analytic Continuation]
  Suppose $f, g : U \to \CC$ are holomorphic on some open connected $U \subseteq \CC$. Suppose $V \subseteq U$ is open and $f(z) = g(z)$ for all $z \in V$ (or, even some sequence of distinct points with limit point in $U$). Then, $f(z) = g(z)$ for all $z \in U$.
\end{corollary}
We can use this theorem to show the existence of or even construct analytic extensions of functions. Suppose we have holomorphic functions $F : U \to \CC$ and $f : V \to \CC$ where $V \subseteq U \subseteq \CC$. If $F = f$ on $V$, then we know that $F$ is a (unique) analytic continuation of $f$ into $U$.

%% Maximum Modulus Principle 
This is a strengthening of the idea that ``a continuous function on a compact set obtains its maximum or minimum'' that is true in $\CC$. First, a lemma:
\begin{lemma}
  Let $U$ be an open neighborhood of a point $z_0 \in \CC$ and $f$ be holomorphic on $U$. Then, if $|f(z_0)| \geq |f(z)|$ for all $z \in U$, then $f$ is constant on $U$.
\end{lemma}

\begin{theorem}[Maximum Modulus Principle]
  Let $f : U \to \CC$ be holomorphic on an open set $U \subseteq \CC$. $f$ has no maximum modulus on $U$. Equivalently, if we have a closed subset $D \subseteq U$, then the maximum modulus of $f$ on $D$ must be achieved on the boundary of $D$.
\end{theorem}

\begin{corollary}[The Fundamental Theorem of Algebra, Again (Osborne)]
  % suppose for sake of contradiction p(z) has no zeroes
  % then 1/p is entire
  % consider a disk of radius R centered at origin, and maximum modulus of 1/p must be on the boundary 
  % polynomials get large in modulus for large R, so as R gets large the modulus of 1/p near the boundary of this disk is zero
  % exists an R for which 1/p(0) is larger than the modulus on the boundary
  % EZ
\end{corollary}


